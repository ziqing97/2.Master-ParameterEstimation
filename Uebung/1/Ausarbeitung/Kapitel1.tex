\chapter{Ausarbeitung}
\section{a}
Der Nährungswert $g_0$ wird mit dem Mittelwert von $\bm{l}$ und $\bm{T}$ berechnet:
\begin{gather*}
	\bar{T} = 2,451 \ut{s} \\
	\bar{l} = 1,480 \ut{m} \\
	g_0 = \frac{\bar{l} \cdot 4 \cdot \pi^2}{\bar{T}^2} = 9,724 \ut{m/s^2}
\end{gather*}
\section{b}
Bedingungsgleichungen:
\begin{equation*}
	T^2 = 4 \cdot \pi^2 \cdot \frac{l}{g}
\end{equation*}
\section{c}
\begin{equation*}
	f(T,l,g) = T^2 - 4 \cdot \pi^2 \cdot \frac{l}{g}
\end{equation*}
Linearisierung
\begin{gather*}
	\frac{\partial f}{\partial T} = 2T \\
	\frac{\partial f}{\partial l} = \frac{4 \pi^2}{g} \\
	\frac{\partial f}{\partial g} = - \frac{4 \pi^2 l}{g^2} \\
	f(\tilde{T},\tilde{l},\tilde{g}) = f(T,l,g)|_0 + \frac{\partial f}{\partial T}(\Delta T + e_T) + \frac{\partial f}{\partial l}(\Delta l + e_l) + \frac{\partial f}{\partial g}(\Delta g)
\end{gather*}
\section{d}
\begin{gather*}
	f(\tilde{T},\tilde{l},\tilde{g}) = \bm{w} + \bm{B^T}\bm{e} + \bm{A}\Delta g \\
	\bm{B^T} = \begin{bmatrix}
		2T_1 & 0 & 0 & \cdots & 0 & \frac{4 \pi^2}{g} & 0 & 0 & 0 \\
		0 & 2T_2 & 0 & \cdots & 0 & 0 & \frac{4 \pi^2}{g} & 0 & 0 \\
		0 & 0 & 2T_3 & \cdots & 0 & 0 & 0 & \frac{4 \pi^2}{g} & 0 \\
		\vdots & \vdots &\vdots &\vdots & 0 & \vdots & \vdots & \vdots & \vdots \\
		0 & 0 & 0 & 0 & 2T_n & 0 & 0 & 0 & \frac{4 \pi^2}{g} \\
	\end{bmatrix}\\
	\bm{A} = \begin{bmatrix}
		 - \frac{4 \pi^2 l_1}{g^2} \\
		 - \frac{4 \pi^2 l_2}{g^2} \\
		 - \frac{4 \pi^2 l_3}{g^2} \\
		 \vdots \\
		 - \frac{4 \pi^2 l_n}{g^2}
	\end{bmatrix}\\
	\bm{w} =  f(T,l,g)|_0 + \bm{B^T}\Delta y \\
	\begin{bmatrix}
		\bm{\lambda} \\
		\Delta g
	\end{bmatrix} = \begin{bmatrix}
	\bm{B^T P^{-1} B} & -\bm{A} \\
	-\bm{A^T} & 0
\end{bmatrix}^{-1}  \begin{bmatrix}
	\bm{w} \\
	0
\end{bmatrix}
\end{gather*}
\section{e,f}
Die Ergebnisse werden mit Iteration gelöst. ($\bm{P}$ ist zuerst eine Einheitmatrix.)
\begin{equation*}
	g_1 = 9.7246 \ut{m/s^2}
\end{equation*}
Varianz:
\begin{equation*}
	\sigma^2_1 = 2.65 \cdot 10^{-14}
\end{equation*}
\section{g}
Die obige Rechnungen wird jetzt mit neue Gewichtmatrix $\bm{P}$ berechnet:
\begin{equation*}
	\bm{P}_2 = diag([2.5\cdot10^{-3},2.5\cdot10^{-3},2.5\cdot10^{-3} \cdots 0.2, 0.2, 0.2\cdots 0.2])
\end{equation*}
man kriegt mit $\bm{P}_2$ die neue Ergebnisse:
\begin{equation*}
	g_2 = 9.7262 \ut{m/s^2}
\end{equation*}
Varianz:
\begin{equation*}
	\sigma^2_2 = 1.54 \cdot 10^{-12}
\end{equation*}
\section{SVD Verfahren}
2 svd Verfahren werden da verwendet:
\subsection{Verfahren 1}
$p$ ist der Einheitvektor der zuschätzen Linie;
\begin{gather}
	\bm{p} = \begin{bmatrix}
		\cos \phi \\
		\sin \phi
	\end{bmatrix}\\
	\bm{M} = \begin{bmatrix}
	\bm{l} &	\bm{T}^2 
	\end{bmatrix}
\end{gather}
Das Ziel ist: $min (\bm{Mp})^2 \rightarrow min(\bm{Mp})^t(\bm{Mp}) \rightarrow min(\bm{p^tM^tMp})$
\\\\
$\bm{M^tM}$ ist normal Matrix $\rightarrow svd(\bm{M^tM}) = \bm{U S U^t}$
\\\\
$min((\bm{p^tU}) \bm{S} (\bm{p^tU})^t) \rightarrow \lambda_1 (\bm{u_1^t p})^2 + \lambda_2 (\bm{u_2^t p})^2$, wobei $\lambda_1$ und $\lambda_2$ Eigenwerte sind. 
\\\\
Annehmen
$(\bm{u_1^t p}) = \cos(\alpha)$ und $(\bm{u_2^t p}) = \sin(\alpha)$, dann haben wir:
\\\\
$\cos^2(\alpha) = s \rightarrow min((\lambda_1 - \lambda_2)s + \lambda_2) \rightarrow s = 1$
\\\\
$\bm{p} = \bm{u_2} = \bm{v_2}$, In der Realität kann man $\bm{V}$ Matrix direkt von $svd(\bm{M})$ berechnet.\\\\ $\frac{4 \pi^2}{g} = \frac{V(1,2)}{V(2,2)} \longrightarrow g = 9.7188 \ut{m/s^2}$

\subsection{Verfahren 2}
\begin{equation*}
	T^2 = 4 \cdot \pi^2 \cdot \frac{l}{g}
\end{equation*}
Diese kann man umschreiben wie:
\begin{equation*}
	\bm{y} = \bm{Ax}
\end{equation*}
wobei $\bm{y} = \bm{T}^2$, $\bm{A} = \bm{l}$ und $\bm{x} = \frac{4 \pi^2}{g}$
\begin{gather*}
	[\bm{U},\bm{S},\bm{V}] = svd(\bm{A}) \\
	g = x_{dach} = \bm{V} \cdot \bm{S}^{-1} \cdot \bm{U}^T \cdot \bm{y} = 9.7232 \ut{m/s^2}
\end{gather*}
Dieses Ergebnis ist ganz gleich wie das von A Modell (normal Least Square)
