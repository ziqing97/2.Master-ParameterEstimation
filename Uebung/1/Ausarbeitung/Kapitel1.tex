\chapter{Ausarbeitung}
\section{a}
Der Nährungswert $g_0$ wird mit dem Mittelwert von $\bm{l}$ und $\bm{T}$ berechnet:
\begin{gather*}
	\bar{T} = 2,451 \ut{s} \\
	\bar{l} = 1,480 \ut{m} \\
	g_0 = \frac{\bar{l} \cdot 4 \cdot \pi^2}{\bar{T}^2} = 9,724 \ut{m/s^2}
\end{gather*}
\section{b}
Bedingungsgleichungen:
\begin{equation*}
	T^2 = 4 \cdot \pi^2 \cdot \frac{l}{g}
\end{equation*}
\section{c}
\begin{equation*}
	f(T,l,g) = T^2 - 4 \cdot \pi^2 \cdot \frac{l}{g}
\end{equation*}
Linearisierung
\begin{gather*}
	\frac{\partial f}{\partial T} = 2T \\
	\frac{\partial f}{\partial l} = \frac{4 \pi^2}{g} \\
	\frac{\partial f}{\partial g} = - \frac{4 \pi^2 l}{g^2} \\
	f(\tilde{T},\tilde{l},\tilde{g}) = f(T,l,g)|_0 + \frac{\partial f}{\partial T}(\Delta T + e_T) + \frac{\partial f}{\partial l}(\Delta l + e_l) + \frac{\partial f}{\partial g}(\Delta g)
\end{gather*}
\section{d}
\begin{gather*}
	f(\tilde{T},\tilde{l},\tilde{g}) = \bm{w} + \bm{B^T}\bm{e} + \bm{A}\Delta g \\
	\bm{B^T} = \begin{bmatrix}
		2T_1 & 0 & 0 & \cdots & 0 & \frac{4 \pi^2}{g} & 0 & 0 & 0 \\
		0 & 2T_2 & 0 & \cdots & 0 & 0 & \frac{4 \pi^2}{g} & 0 & 0 \\
		0 & 0 & 2T_3 & \cdots & 0 & 0 & 0 & \frac{4 \pi^2}{g} & 0 \\
		\vdots & \vdots &\vdots &\vdots & 0 & \vdots & \vdots & \vdots & \vdots \\
		0 & 0 & 0 & 0 & 2T_n & 0 & 0 & 0 & \frac{4 \pi^2}{g} \\
	\end{bmatrix}\\
	\bm{A} = \begin{bmatrix}
		 - \frac{4 \pi^2 l_1}{g^2} \\
		 - \frac{4 \pi^2 l_2}{g^2} \\
		 - \frac{4 \pi^2 l_3}{g^2} \\
		 \vdots \\
		 - \frac{4 \pi^2 l_n}{g^2}
	\end{bmatrix}\\
	\bm{w} =  f(T,l,g)|_0 + \bm{B^T}\Delta y \\
	\begin{bmatrix}
		\bm{\lambda} \\
		\Delta g
	\end{bmatrix} = \begin{bmatrix}
	\bm{B^T P^{-1} B} & -\bm{A} \\
	-\bm{A^T} & 0
\end{bmatrix}^{-1}  \begin{bmatrix}
	\bm{w} \\
	0
\end{bmatrix}
\end{gather*}
\section{e,f}
Die Ergebnisse werden mit Iteration gelöst. ($\bm{P}$ ist zuerst eine Einheitmatrix.)
\begin{equation*}
	g_1 = 9.7246 \ut{m/s^2}
\end{equation*}
Varianz:
\begin{equation*}
	\sigma^2_1 = 2.65 \cdot 10^{-14}
\end{equation*}
\section{g}
Die obige Rechnungen wird jetzt mit neue Gewichtmatrix $\bm{P}$ berechnet:
\begin{equation*}
	\bm{P}_2 = diag([2.5\cdot10^{-3},2.5\cdot10^{-3},2.5\cdot10^{-3} \cdots 0.2, 0.2, 0.2\cdots 0.2])
\end{equation*}
man kriegt mit $\bm{P}_2$ die neue Ergebnisse:
\begin{equation*}
	g_2 = 9.7262 \ut{m/s^2}
\end{equation*}
Varianz:
\begin{equation*}
	\sigma^2_2 = 1.54 \cdot 10^{-12}
\end{equation*}